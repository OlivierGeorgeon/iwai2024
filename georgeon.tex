% This is samplepaper.tex, a sample chapter demonstrating the
% LLNCS macro package for Springer Computer Science proceedings;
% Version 2.21 of 2022/01/12
%
\documentclass[runningheads]{llncs}
%
\usepackage[T1]{fontenc}
% T1 fonts will be used to generate the final print and online PDFs,
% so please use T1 fonts in your manuscript whenever possible.
% Other font encondings may result in incorrect characters.
%
\usepackage{graphicx}
% Used for displaying a sample figure. If possible, figure files should
% be included in EPS format.
%
% If you use the hyperref package, please uncomment the following two lines
% to display URLs in blue roman font according to Springer's eBook style:
%\usepackage{color}
%\renewcommand\UrlFont{\color{blue}\rmfamily}
%\urlstyle{rm}
%
\begin{document}
%
\title{Reducing Prediction Error by Refining the Game Engine In The Head}
%
\titlerunning{GEITH Refinement}
% If the paper title is too long for the running head, you can set
% an abbreviated paper title here
%
\author{Olivier L. Georgeon\inst{1}\orcidID{0000-1111-2222-3333} \and
Second Author\inst{2,3}\orcidID{1111-2222-3333-4444} \and
Third Author\inst{3}\orcidID{2222--3333-4444-5555}}
%
\authorrunning{O. Georgeon et al.}
% First names are abbreviated in the running head.
% If there are more than two authors, 'et al.' is used.
%
\institute{UR CONFLUENCE: Sciences et Humanites (EA 1598), UCLy, France 
	\email{ogeorgeon@univ-catholyon.fr} \and
Springer Heidelberg, Tiergartenstr. 17, 69121 Heidelberg, Germany
\email{lncs@springer.com}\\
\url{http://www.springer.com/gp/computer-science/lncs} \and
ABC Institute, Rupert-Karls-University Heidelberg, Heidelberg, Germany\\
\email{\{abc,lncs\}@uni-heidelberg.de}}
%
\maketitle              % typeset the header of the contribution
%
\begin{abstract}
We present a method for a robot to generate behaviors to refine its ``Game Engine in the Head'' (GEITH).
The behaviors result in posterior sensory signals that carry information to refine the GEITH parameters. 
At the beginning of each interaction cycle, the robot uses its GEITH to run a simulation to compute predicted sensory signals. 
For each sensor, prediction error is the difference of the predicted sensory signal minus the actual sensory signal received at the end of the interaction cycle. 
Results show that over a few hundred interaction cycles, the robot manages to satisfactorily calibrate its GEITH, and prediction errors decrease. 
Moreover, the robot generates behaviors that human observers describe as playful.

\keywords{Active infernce  \and constructivist learning \and enaction \and intrinsic motivation \and robotics.}
\end{abstract}
%
%
%
\section{Introduction}

It is widely believed that cognitive beings possess some kind of a \textit{world model} that they use to generate intelligent behaviors.
How they construct and maintain this world model remains, however,  an open question in cognitive science and artificial intelligence. 

Karl Friston and his research group have proposed Active Inference \cite[e.g.]{smith_step-by-step_2022} as a method to infer the world model by minimizing free energy \cite{friston_free-energy_2010}.
The world model is represented as a probability distribution $\mu = P(S)$ over the set $S$ of possible world states. 
This method iteratively updates the world model after interaction through gradient descent of free energy. 
The variational free energy amounts to the divergence between two probability distributions: the estimated world model $\mu$ and the joint probability distribution of observations and world states called the \textit{generative model}. 
This method, however, requires that the set of states $S$ and the relations between states and observations be known \textit{a prior}. 
Moreover, the high computational needs to compute the free energy and the high number of interaction cycles to converge to a useful world model makes this method inapplicable in our case of a robot interacting with the open world. 

The Partially Observable Markov Decision Process (POMDP) literature proposes a broad range of methods to infer a \textit{belief state} in a partially observable process.
The belief state amounts to the agent's world model of the environment that the agent can only partially observe.  
If the state transition function and the observation function are known \textit{a priori}, the problem of computing the belief state has been mathematically solved \cite{astrom1965optimal}. 
It was also proven that the implementation of the solution becomes intractable as the set of states and observation grows. 
In the absence of these presupposition, the problem of inferring belief states in POMDPs does not lend itself to a mathematical analysis. 

The active inference and the POMDP literature suggests that inferring the world model through experience of interaction requires prior assumptions to reduce complexity. 
The present study proposes the hypothesis that the ``Game Engine In The head'' (GEITH) can work as a suitable prior assumption. 

Joshua Tenenbaum and his research group have proposed the GEITH \cite{battaglia_simulation_2013} as the capacity of cognitive beings to simulate basic dynamics of physics and interactions. 
In mammals, the GEITH would rest upon brain structures that are partially predefined by genes and then completed through ontogenetic development.  
Similarly, it is possible to endow artificial agents and robots with a predefined software game engine, and expect them to refine the parameters of their game engine as they test their predictions in the world.

The refinement of the game engine is measured through two methods. 
The first is performed by the robot itself by measuring the prediction error of sensory signals. 
Decrease in prediction errors show improvement of the game engine. 
The second is performed by the experimenter by assessing whether the game engine parameters converge towards a target range that indicates that the robot managed to calibrate its GEITH. 

fqfdqsfdqs
fdsqfdsq
fdqs

fdqsfdqs

fdqs


fqsfqs

fdqsfdqs

fdqsfqs


\section{Experimental setup}

\subsection{The robotics platform}

\subsection{The Game Engine In The Head}

\section{Behavior generation}

\section{Results}

\section{Conclusion}



\section{First Section}
\subsection{A Subsection Sample}
Please note that the first paragraph of a section or subsection is
not indented. The first paragraph that follows a table, figure,
equation etc. does not need an indent, either.

Subsequent paragraphs, however, are indented.

\subsubsection{Sample Heading (Third Level)} Only two levels of
headings should be numbered. Lower level headings remain unnumbered;
they are formatted as run-in headings.

\paragraph{Sample Heading (Fourth Level)}
The contribution should contain no more than four levels of
headings. Table~\ref{tab1} gives a summary of all heading levels.

\begin{table}
\caption{Table captions should be placed above the
tables.}\label{tab1}
\begin{tabular}{|l|l|l|}
\hline
Heading level &  Example & Font size and style\\
\hline
Title (centered) &  {\Large\bfseries Lecture Notes} & 14 point, bold\\
1st-level heading &  {\large\bfseries 1 Introduction} & 12 point, bold\\
2nd-level heading & {\bfseries 2.1 Printing Area} & 10 point, bold\\
3rd-level heading & {\bfseries Run-in Heading in Bold.} Text follows & 10 point, bold\\
4th-level heading & {\itshape Lowest Level Heading.} Text follows & 10 point, italic\\
\hline
\end{tabular}
\end{table}


\noindent Displayed equations are centered and set on a separate
line.
\begin{equation}
x + y = z
\end{equation}
Please try to avoid rasterized images for line-art diagrams and
schemas. Whenever possible, use vector graphics instead (see
Fig.~\ref{fig1}).

\begin{figure}
\includegraphics[width=\textwidth]{fig1.eps}
\caption{A figure caption is always placed below the illustration.
Please note that short captions are centered, while long ones are
justified by the macro package automatically.} \label{fig1}
\end{figure}

\begin{theorem}
This is a sample theorem. The run-in heading is set in bold, while
the following text appears in italics. Definitions, lemmas,
propositions, and corollaries are styled the same way.
\end{theorem}
%
% the environments 'definition', 'lemma', 'proposition', 'corollary',
% 'remark', and 'example' are defined in the LLNCS documentclass as well.
%
\begin{proof}
Proofs, examples, and remarks have the initial word in italics,
while the following text appears in normal font.
\end{proof}
For citations of references, we prefer the use of square brackets
and consecutive numbers. Citations using labels or the author/year
convention are also acceptable. The following bibliography provides
a sample reference list with entries for journal
articles~\cite{ref_article1}, an LNCS chapter~\cite{ref_lncs1}, a
book~\cite{ref_book1}, proceedings without editors~\cite{ref_proc1},
and a homepage~\cite{ref_url1}. Multiple citations are grouped
\cite{ref_article1,ref_lncs1,ref_book1},
\cite{ref_article1,ref_book1,ref_proc1,ref_url1}.

\begin{credits}
\subsubsection{\ackname} A bold run-in heading in small font size at the end of the paper is
used for general acknowledgments, for example: This study was funded
by X (grant number Y).

\subsubsection{\discintname}
It is now necessary to declare any competing interests or to specifically
state that the authors have no competing interests. Please place the
statement with a bold run-in heading in small font size beneath the
(optional) acknowledgments\footnote{If EquinOCS, our proceedings submission
system, is used, then the disclaimer can be provided directly in the system.},
for example: The authors have no competing interests to declare that are
relevant to the content of this article. Or: Author A has received research
grants from Company W. Author B has received a speaker honorarium from
Company X and owns stock in Company Y. Author C is a member of committee Z.
\end{credits}
%
% ---- Bibliography ----
%
% BibTeX users should specify bibliography style 'splncs04'.
% References will then be sorted and formatted in the correct style.
%
\bibliographystyle{splncs04}
\bibliography{georgeon.bib}
%
\end{document}
